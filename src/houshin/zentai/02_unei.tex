\subsection*{運営方針}

%\writtenBy{\president}{斎藤}{竜也}
\writtenBy{\subPresident}{齋藤}{竜也}
%\writtenBy{\firstGrade}{斎藤}{竜也}
%\writtenBy{\secondGrade}{斎藤}{竜也}
%\writtenBy{\thirdGrade}{斎藤}{竜也}
%\writtenBy{\fourthGrade}{斎藤}{竜也}

2021年度秋学期の運営に関して以下の5点から方針を述べる.
\begin{itemize}
    \item 定例会議
    \item 上回生会議
    \item 局
    \item 企画
    \item 運営サポート
\end{itemize}

\subsubsection*{定例会議}
春学期同様木曜日に開催する.
必要に応じてSlackのチャンネルで資料や議題の周知と共有を行う.
部室利用の申請は承諾されているものの対面での集合は現実的ではないため,
秋学期も基本的にはリモートでの開催を予定している.

\subsubsection*{上回生会議}
秋学期も春学期同様に毎週開催する.
各局の局長及びその代理人が必ず参加するものとする.
また,例年通り議決権のない会員,特に\firstGrade{}に対して上回生会議に
参加可能である旨を告知する.

\subsubsection*{局}
局配属に関しては春学期中に完了したため,本セッションでは局会議についてのみ言及する.
局会議では2020年度より毎週開催されていない問題が続いていた.これは議題の多さと会議の頻度が
釣り合わないことによるものであると考えられるため,2021年度秋学期では局会議の定期的な開催の強制をしないこととする.
ただし,上回生会議にて都度局での状況を確認し,必要に応じて局会議を開催する.

\subsubsection*{企画}
担当者は常に会員2名以上で対応することが望ましい.担当者の一人が退会などの理由で対応できなくなった場合,
新たに担当者を追加するか,2020年度担当者がフォローできるようにする.
KPTに関しては,春学期同様上回生会議で行うものとする.

\subsubsection*{運営サポート}
春学期同様,\thirdGrade{}がサポートするが,春学期の状況などを顧み,より\secondGrade{}中心の運営を行なっていく.
秋学期からは\thirdGrade{}が研究室に配属されるため,必要な時は可能な限り早めの連絡を心がける.
