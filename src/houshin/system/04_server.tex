\subsection*{サーバ管理方針}

%\writtenBy{\systemChief}{宇佐}{基史}
\writtenBy{\systemStaff}{宇佐}{基史}

サーバ管理方針として以下の五つを方針とする.
\begin{itemize}
    \item サーバに用いるOSの選定
    \item 各種ミドルウェアの更新
    \item NASの運用
    \item 管理・運用の属人化を防ぐ
    \item 自主ドメインの更新など
\end{itemize}

\subsubsection*{サーバに用いるOSの選定}
Cent Streamの有用性が本会サーバに求められる水準を満たしていないことを受け,
代替となるサーバ用のOSを引き続き調査していく.
アップデート実施の是非,各種依存環境の調査,
情報理工学部の大阪いばらきキャンパス移転に伴う本会の移設との兼ね合いなどを視野に入れ,
中長期的な視点で更新を行える製品を採用する.
適切な対応が決定されるまでは現行のCentOS 7を引き続き用いることとする.

\subsubsection*{各種ミドルウェアの更新}
本局会議内における脆弱性情報の確認,
年に複数回の最新バージョン確認を引き続き行っていく.
必要に応じてミドルウェアの更新を行う.
更新作業に際して生じた問題は解決の過程とともに適宜記録をし,
今後同様の問題が生じた場合の参考となるように努める.

\subsubsection*{NASの運用}
NASの運用に伴う方針決定と実務を行う.
本会が今後行うあらゆる情報保存の統合を最終的な目的とする.
本会活動のあらゆる引き続き業務に活用しうる運用を目指す.
また,NASの新規導入以前にはWindowsの同時ログイン時におこる不具合や各操作環境からくる動作の不安定性といった問題が存在したが,
これらプロファイル問題の解決も目指す.
サークルルーム外からのアクセスも視野に運用方針を立てる.
コロナ禍においても可能な範囲で実務,実地運用を行っていく.

\subsubsection*{管理・運用の属人化を防ぐ}
サーバ管理またはサーバ運用における属人性を解決していく.
引き継ぎも適切に行い,継続して属人化を防ぐ取り組みを行っていく.
保存媒体を明確にし,2021年度秋学期中のシステム管理局引き続き資料完成を目指す.

\subsubsection*{自主ドメインの更新など}
引き続き自主ドメインの更新を行う.
例年と同様,停電対応時の作業と同時に行うこととする.
更新作業に際して生じた問題は解決の過程とともに適宜記録をし,
今後同様の問題が生じた場合の参考となるように努める.
