\subsection*{全体総括}

%\writtenBy{\kensuiChief}{八木田}{裕伍}
\writtenBy{\kensuiStaff}{八木田}{裕伍}


本項では本局におけるプロジェクト活動業務に関する2021年度春学期の総括を以下の点において述べる.

\begin{itemize}
\item 企画書の募集
\item 週報の回収・催促
\item 会員のプロジェクト管理
\item 発表の機会の提供
\item 報告書の管理
\end{itemize}

\subsubsection*{企画書の募集}

プロジェクト活動の企画書は最初の募集で三つ,追加募集で二つ提出され,最終的に五つ企画書が提出された.
企画書を局会議と上回生会議で確認を行い,全ての企画書に問題が無かったため,全てのプロジェクトが設立された.

\subsubsection*{週報の回収・催促}

各プロジェクトリーダーは,プロジェクト活動の進捗確認や問題の有無の確認を行うために,
週報の提出を義務付けられている.
週報の回収にはGoogleフォームが用いられ,Slack のリマインダー機能を用いてリマインドが行われた.
週報の提出が遅延したプロジェクトは無かった.

\subsubsection*{会員のプロジェクト管理}

本局は,各会員がどのプロジェクトに所属しているかを把握し,
プロジェクトが途中で終了した場合などに所属していた会員のプロジェクト異動などを管理している.
プロジェクト間での会員の異動はなかった.

\subsubsection*{発表の機会の提供}

プロジェクト活動の成果発表をプロジェクト発表会を通じて行った.
発表時間を予め制限していたため,発表は円滑に進行した.
