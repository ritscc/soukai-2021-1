\subsection*{全体総括}

%\writtenBy{\kensuiChief}{八木田}{裕伍}
\writtenBy{\kensuiStaff}{八木田}{裕伍}

2021年度春学期の研究推進局は以下の3点を目的として活動を行った.

\begin{itemize}
    \item 平常活動の支援
    \item 会員が興味関心のある活動ができる環境づくり
    \item 発信力を養うための環境づくり
\end{itemize}

\subsubsection*{平常活動の支援}
平常活動の支援に関しては,プロジェクト活動の進捗管理やサポート,追い込み合宿,プロジェクト発表会の準備と進行を行った.

プロジェクト活動の進捗管理では,週報を用いてプロジェクト活動の進捗確認や問題の有無の確認を行い,
問題が確認された場合は,それを上回生会議の議題に上げることで問題の解決を図るという方針であったが,
2021年度春学期はプロジェクト活動中に問題が発生しなかった.
また,Slack のリマインダー機能を使用しプロジェクトリーダーへ週報の提出を促していたため,
ほとんどの班は遅延なく週報を提出することができた.

2020年度秋学期と同様,2021年度春学期は,
プロジェクト活動,追い込み合宿,プロジェクト発表会はオンラインのみであり,
部屋取りは行わなかった.

追い込み合宿の進行は,開始や報告書のアナウンスを行った.

プロジェクト発表会では,進行と発表の録画を行った.
予め発表時間の目安を決定していたため,長時間や短時間の発表は無く,円滑に進行できた.

\subsubsection*{会員が興味関心のある活動ができる環境づくり}
会員が興味関心のある活動ができる環境づくりに関しては,
夏期勉強会の準備を行った.

夏期勉強会の準備は,自分が開催したい勉強会と,他の会員に開催してほしい勉強会を募集した.
全ての勉強会はオンラインで開催されたため,部屋取りは行わず,開催日時や形式は開催者に任せることとした.

2021年度は,以下に示す 4 つの勉強会が開催されたか,開催予定である.

\begin{itemize}
    \item React勉強会
    \item DTM勉強会
    \item TDD勉強会
    \item バックエンド勉強会
\end{itemize}

\subsubsection*{発信力を養うための環境づくり}
発信力を養うための環境づくりに関しては,LTとプロジェクト発表会の運営を行った.

LTは,毎週の定例会議中で行われた.
2021年度春学期は,LTを担当週までに行うことができず,遅延して行われたものがあり,最終週に発表が集中する事態が発生した.
理由としては,担当者のアナウンスがメールのみであり,毎回の LT 後のアナウンスや,Slack でのアナウンスを行っていなかったことが挙げられる.

また,LT意欲向上のため,LTアンケートを行い,入賞者には本会で購入する本の選択権を与えた.
2021年度春学期は,LTアンケートの実施から結果の公表まで円滑に実施することが出来た.
これは,研究推進局が定例会議などのスケジュールを把握して事前に行動していたためである.
