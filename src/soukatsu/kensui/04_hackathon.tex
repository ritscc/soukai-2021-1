\subsection*{ハッカソン総括}


\writtenBy{\kensuiChief}{Park}{Jooinh}
%\writtenBy{\kensuiStaff}{Park}{Jooinh}

ハッカソン総括にあたり,2021年度に行われたハッカソンは夏期ハッカソンのみであったため,主に扱う内容は夏期ハッカソンのことである.

\subsubsection*{全体総括}

2021年度夏期ハッカソンは夏期休暇中の8月30日から9月1日の3日間開催された.参加者は14名であったため,4〜5名からなる3グループでの制作となった.

ハッカソン開催の目的及び目標として掲げていた制作活動の促進や技術力・発表力の向上,会員間の交流を深める
といったような点は達成できたが,参加人数の低迷は憂慮すべき点であると考えられる.特に\firstGrade{}と\secondGrade{}の参加人数が少なかった.

また,今回は株式会社ジースタイラス様の協賛の下で行われたが,渉外局との連携により,運営を円滑に進めることが
できたと考えられる.

\subsubsection*{事前運営総括}

企画の告知においては,2021年度春学期が終了した後にハッカソンの開催が決まったため定例会議時にて告知できず,
Slackでのみ告知とリマインドを行った.

2021年度夏期ハッカソンはオンライン開催であったため,例年行っていたエポック立命21の予約などは特に行わなかった.

開発ツールとしてDiscordとSlackを使い,開会式と閉会式はZoomを介して行った.
テーマに関しては開会式当日に発表し,アイデアソンを行った.

\subsubsection*{当日運営総括}

例年に比べて参加者の遅刻は非常に少なく,企画書通りのスケジュールで進行することができた.

成果物発表では,全てのチームが発表スライドを作成し,Zoomを使用して発表を行った.
その際,株式会社ジースタイラス様の社員の方に各グループの発表に対する講評頂いた.
