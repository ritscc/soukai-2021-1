\subsection*{\newGradeIfKouki{}\thirdGrade{}方針}

\writtenBy{\thirdGrade}{桐井}{優実}

3回生は以下の3点を方針としていた.

^\begin{enumerate}
    \item 基本的には2回生に運営を任せ,3回生は対面行事に関して強くサポートする
    \item 活動に積極的に参加する
    \item 適度に活動しやすい雰囲気づくりを行う
\end{enumerate}


1については,2021年度は2020年度に比べて多くのイベントを復活することができたこともあり,
3回生のサポートのもと2回生を中心とした行事の運営を行うことができた.
対面行事は未だに行えていないものの,2回生のサポートをするという点に関しては達成できた.
また通常時の運営に関しては,例会終了後にZoomを解放することでフランクな雰囲気の中で交流する場を設けることが
できた.

2については,3回生がリーダーを務めるプロジェクトが2つ建てられたり,3回生の夏ハッカソンの参加率が高かったりしたことから
概ね達成できたと言える.

3については,三回生主催の勉強会を開催することで,下回生の開発を促進することができた.
しかし,まだ勉強会の新入生の参加者が少ないため対応策として新入生にもっとSlackを見てもらうための工夫
などが必要であると考える.

その他,2020年度に達成しきれなかったものに関して記述する.
まず局配属の時期については,夏季休暇中総会前にZoomを利用して新局員との顔合わせを行うことができた.
遅刻・欠席連絡については,事後報告はあったもののSlackの#absent_lateが活用され,無断欠席者はいなかった.

