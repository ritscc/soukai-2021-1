\subsection*{Welcomeゼミ総括}

%\writtenBy{\president}{田尻}{聖奈}
%\writtenBy{\subPresident}{田尻}{聖奈}
%\writtenBy{\firstGrade}{田尻}{聖奈}
\writtenBy{\secondGrade}{田尻}{聖奈}
%\writtenBy{\thirdGrade}{田尻}{聖奈}
%\writtenBy{\fourthGrade}{田尻}{聖奈}

2021年度も2020年度に引き続きWelcomeゼミを行った。
形式は2020年度同様、新入生の希望に応じて上回生との1対1の指導方式で行った。

上回生から担当できる分野を聞き出し、新入生には希望する分野を選んでもらい、ペアを組んだ。
DTMを志望している新入生が多く、希望分野に偏りがあった。
二回生主体で運営するはずだったが、新入生の希望した分野を担当できる二回生が少なく、三回生を割り振ったペアが数例あった。
担当上回生のWelcomeゼミ進捗についての報告はあまり見られなかった。

また、2021年度は以下を目標として進行した。
\begin{itemize}
	\item 新入生にとってRCCの部室が居心地のよい空間にする
	\item 気軽に部室に来てもらう
	\item 新入生の中長期的な定着
\end{itemize}

オンライン活動であったため、部室関係の目標は未達成だが、入部率は高く、新入生の中長期的な定着目標は達成できたといえる。
成果物発表会は5人が発表した。原則発表であったが、参加が遅かった新入生などは発表できなかった。