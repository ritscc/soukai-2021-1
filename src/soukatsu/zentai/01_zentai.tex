\subsection*{2021年度春学期活動総括}

%\writtenBy{\president}{深田}{紘希}

本会の目的である「情報科学の研究,及びその成果の発表を活動の基本に会員相互の親睦を図り,学術文化の創造と発展に寄与する」ことを達成するため,方針として以下の六つを立てた.
これらについてそれぞれ評価を行うことで2021年度春学期の総括とする.

\begin{itemize}
    \item 親睦を深める
    \item 規律ある行動
    \item 自己発信力の向上
    \item 会員間の技術向上
    \item 外部への情報発信
    \item 持続可能な運営
\end{itemize}

\subsubsection*{親睦を深める}
    2021年度春学期活動では,Welcomeゼミや新歓交流会,プロジェクト活動を実施することによって会員間の親睦を図った.

    尚,2021年度春学期活動においては何れの活動も,コロナウイルス感染拡大防止のためオンライン形式で行った.

    Welcomeゼミは,新入生と上回生が親睦を深める重要な機会であった.
    オンライン形式での実施となったため,直接ペアを組んでいない新入生と上回生が親睦を深めることは困難であったが,
    対面での活動ができない中で充分親睦を深めることができた.

    新歓交流会にも多くの新入生が参加し,自己紹介後には交流の機会も設けた.
    上回生との会話を通して,良い影響があったように思われる.

    プロジェクト活動では,各々の班で共同開発や発表が活発に行われていたが,
    オンライン形式である都合上,会員同士の交流の機会はやや少なかった.

\subsubsection*{規律ある行動}
    本項では,遅刻・欠席連絡と備品整備,サークルルームの使用方法の三つについて評価する.

    遅刻・欠席連絡については,概ねSlackの専用チャンネルにおいて行われていた.例年に比べ連絡頻度は多い一方,下回生の模範となるべき上回生は,特定の人のみが連絡しているのが実情である.また開始時刻を過ぎてからの連絡が少なからず見受けられた.

    備品整理及びサークルルームの使用方法については,コロナウイルス感染拡大防止のためサークルルームが使用不可であったため,備品の使用や入室ができず,問題は発生しなかった.

\subsubsection*{自己発信力の向上}
    自己発信力の向上の機会として,2021年度春学期活動では,LTを行った.

    LTは,割り当てられていた会員の全員が発表したため,定例会議での発表は充実していた.
    しかし新入生の飛び入りLTは無く,また有志の飛び入りも多いとはいえなかった.積極的な会員のみが発表を行う状態であった.

    プロジェクト発表会については,設立された全てのプロジェクトが報告書執筆とその発表を行った.
    その他,行事ごとの発表資料や文書に対する会員間のレビューは積極的に行われた.

\subsubsection*{会員間の技術向上}
    会全体の技術力を向上させることを目的として,LTやプロジェクト活動,勉強会を開催した.

    LTでは会員が自身の興味対象について深い内容を扱ったため,伝わりづらい点もあったが,会員の技術力に良い影響を与えていたと考えられる.
    しかし新入生の発表は無く,また有志の飛び入りも多いとはいえなかった.

    プロジェクト活動や勉強会も,オンライン形式で問題なく行われた.

\subsubsection*{外部への情報発信}
    会外へ活動を発信する機会として,主に本会Webサイトと会公式Twitter,KC3が挙げられる.

    本会Webサイトでは,2021年度春学期プロジェクト活動報告書を掲載した.
    本会Webサイトの更新は2021年度春学期プロジェクト活動報告書のみであり,不足していた.新歓に関する告知やイベントに関しての記事を一切上げることができなかった.
    
    会公式Twitterでは,LTやイベントが行われる度にその様子が発信された.
    こちらは頻度が十分であり,内容も適切であった.
    
    また,KC3では本会の活動紹介や勉強会開催を通して,本会の活動を発信した.
    懇親会においては,複数の会員が参加し,自身の興味分野について発信した.
    
\subsubsection*{持続可能な運営}
    本会はコロナ禍によって,多くの活動が縮小を迫られた.
    そこで持続可能な運営にすべくサークルルームの利用再開,新入生の勧誘と中止イベントの再開を挙げた.
    
    まずサークルルームの利用再開については,
    立命館大学のガイドラインに沿いつつ本会の実情に合わせた部の対応方針を作成し,学生部に提出した.
    ここで複数回の再提出を経て,秋学期から制限付きでのサークルルーム利用再開を許可された.
    よってこの目標は一部達成することができた.
    
    次に新入生の勧誘について,2020年度に実施されていたWeb新歓にフォーカスしていたが,2021年度は開催されなかった.
    ポスター掲示は行ったものの,大きな成果を得られたとは言い難い.
    ブース出展に関しては5月6月前半分は提出したものの実施されず,6月後半の部については連絡不足により提出することができなかった.
    このような状況であったため,本来入会を希望したであろう新入生を十分確保できたとは言えず,目標が達成されたとは言えない.
    
    次にWeb上でのイベント再開を目標とした.2021年度に再開しなければ引き継ぎが途絶えてしまうためである.
    中止イベントの再開として,2021年度は夏ハッカソンと夏勉強会と立命の家を実施し,夏期成果物発表会を計画した.
    春学期のイベントは,コロナ禍以前の2019年度に行っていたイベントすべてを再開・計画することができた.
    よって十分目標を達成することができた.
    