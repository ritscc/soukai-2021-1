\subsection*{運営総括}

%\writtenBy{\president}{斎藤}{竜也}
\writtenBy{\subPresident}{齋藤}{竜也}
%\writtenBy{\firstGrade}{斎藤}{竜也}
%\writtenBy{\secondGrade}{斎藤}{竜也}
%\writtenBy{\thirdGrade}{斎藤}{竜也}
%\writtenBy{\fourthGrade}{斎藤}{竜也}

2021年度春学期の運営を以下の4点から述べる.
\begin{itemize}
    \item 定例会議
    \item 上回生会議
    \item 局
    \item 企画
\end{itemize}

\subsubsection*{定例会議}
毎週木曜日にZoomを用いて開催した.内容は例年通り執行部及び局からの連絡,会員によるLTであった.
2020年度はZoomミーティングルームの準備を執行部が行っていたが,総務局に一任した.
これによる不具合などはなく円滑な準備が行えていた.

Slackの議題共有チャンネルは用いられることはなかったが,上回生会議での議題共有で十分であった.
本会のフォームを管理しているform viewerの積極的な活用を方針にしていたが,前半は引き継ぎが十分でなく,独自にフォームを作成していた.
後半では,十分活用できていたため問題はないと思われる.

また,遅刻・欠席連絡では特定の会員のみが連絡をしていた.

しかし参加率などは2020年度と変わりなかった.また,入会希望の新入生には積極的にZoomに参加してもらうように対応し,定例会議に参加しやすいようにした.

総評としては,定例会議の運用で大きな問題が起こることなく活動できていたと言える.

\subsubsection*{上回生会議}
毎週水曜日にDiscordを用いて開催した.
参加に関しては大きな問題はなかった.
企画担当者やプロジェクトリーダーの招集などの出席も問題なく参加できており,
2020年度の執行部もフォローのため参加していた.

議題の共有など十分できており,上回生のフォローもあり運営は円滑に行われた.

\subsubsection*{局}
局の運営では局会議と局配属にさらに分けて述べる.
\paragraph*{局会議}
週1回の開催を方針にしていたが,多くの局が実施できなかった.
リモートでの活動が余儀なくされ,議題の少なさから局会議の開催に関して度々問題になっていたため,
開催方法について一度考え直す必要がある.
\paragraph*{局配属}
2020年度の反省を踏まえ,夏期休暇前に配属を行う方針をたてていた.多少の遅れがあったものの局話し合いまでに配属を行うことができたが,
新入生の回生を把握しきれておらず一部局員を異動させなければならない事態が発生した.

総評としては,配属などコミュニケーションエラーが発生していることなど見受けられた.

\subsubsection*{企画}
企画担当者の上回生会議への出席は前述のように問題なかった.新歓など2020年度を踏襲し,リモートでの開催を前提に準備していたが,
対面での新歓開催になりその対応に遅れが出てしまうことがあった.
KPTに関しては上回生会議で行うこととしたため大半の企画で問題なく行うことができた.
上回生がフォローに入りつつも企画運営の主体は2回生にあったため企画の引き継ぎに関しては問題なく行えていたと言えるだろう.

総評としては進捗確認などは問題なく行えていたが,学生部などとの連絡が円滑に行えていなかったため,
執行委員長宛のみのメールではなく執行部のメーリングリストに変更を申し出るなどして対応していく.