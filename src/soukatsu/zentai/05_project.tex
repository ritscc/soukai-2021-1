\subsection*{プロジェクト活動総括}

%\writtenBy{\president}{Park}{Jooinh}
%\writtenBy{\subPresident}{Park}{Jooinh}
%\writtenBy{\firstGrade}{Park}{Jooinh}
\writtenBy{\secondGrade}{Park}{Jooinh}
%\writtenBy{\thirdGrade}{Park}{Jooinh}
%\writtenBy{\fourthGrade}{Park}{Jooinh}

\subsubsection*{全体総括}
2021年度春学期のプロジェクト活動は,5月上旬頃から企画書の募集を開始し,6月中旬に活動を開始した.
各プロジェクトには活動ごとに週報を提出することを義務付け,進捗確認を行った.
全5個のプロジェクト全てが設立された.
2021年度春学期も2020年度秋学期に続き新型コロナウイルス感染症流行のため,
プロジェクト活動,追い込み合宿,プロジェクト発表会を全てオンラインでの開催とした.
2021年度は半期プロジェクトと通年プロジェクトのどちらも設立し,
プロジェクト発表会で通年のプロジェクトは中間報告書と途中成果を発表することとした.

以下に2021年度春学期に活動していたプロジェクトの一覧を示す.

通年プロジェクト
\begin{itemize}
  \item DTM班
  \item OS班
\end{itemize}

半期プロジェクト
\begin{itemize}
  \item LTプレゼン班
  \item Web班
  \item スロット班
\end{itemize}

プロジェクト活動の総括は以下の六つに分けて行う.

\begin{itemize}
  \item 目標の総括
  \item プロジェクトの内容
  \item 週報
  \item 報告書
  \item 追い込み合宿
  \item プロジェクト発表会
\end{itemize}

\subsubsection*{目標の総括}
2021年度春学期の目標は以下の三つであった.

\begin{itemize}
  \item 活動を通して技術力の向上を図る
  \item 個人のみならずグループ活動としての経験を得る
  \item 活動によって得られた成果を本会Webサイトを通して公開する
\end{itemize}

これらを踏まえた総括を以下に記す.

活動を通して技術力の向上を図るに関しては,
プロジェクト活動をオンラインのみでの活動に限定していたにも関わらずプロジェクト進行の遅延はなかった.
新入生の中にはプロジェクトに参加していなかった者が数人見受けられたが,
ほぼ会員の技術力が向上したことから,この目標は概ね達成できたと言える.

集団行動の重要性を学ぶに関しては,
全般的には毎回のプロジェクト活動に参加率は高く,達成できていたと言える.

得られた成果を本会Webサイトを通して公開するに関しては,
全てのプロジェクト班が報告書を問題なく提出したことを確認し,
これらをWebサイトを通して公開したため,達成できたと言える.

\subsubsection*{プロジェクトの内容}
プロジェクトの内容については,全ての班において適切であった.

\subsubsection*{週報}
2020年度秋学期同様,2021年度春学期も研究推進局内のみではなく,上回生会議においても週報の確認を行った.
週報の内容に問題はなく,提出の遅れもなかったため,週報の管理は問題なくできていたと言える.

\subsubsection*{報告書}
半期プロジェクトは報告書の提出をもってプロジェクト終了とし,通年プロジェクトは途中成果を示すこととする.
報告書の必須項目を以下に示す.

\begin{itemize}
  \item 活動動機,目的
  \item 活動内容
  \item 活動結果
  \item 考察
  \item 参考文献
\end{itemize}

報告書の内容に関しては,ほぼ問題はなかった.
執筆に関わっていない班員が一定数見られたが,全般的な執筆率は高く,提出が遅れることもなかった

\subsubsection*{追い込み合宿}
8月5日,6日にプロジェクト発表会に向けて班員に準備を行ってもらうために,オンラインでの追い込み合宿を行った.
原則全員参加であるが,準備が完了している班に関しては参加しなくても良いものとした.

事前にSlackを通して開催日時を告知し,当日は各プロジェクトリーダーを中心にDiscordにて作業を行った.

追い込み合宿に参加した会員の大半は,追い込み合宿で進捗を出していたことから
追い込み合宿は十分な有用性があると言える.

\subsubsection*{プロジェクト発表会}
プロジェクト活動を通して得ることができた知見や技術を会内で共有する場として,オンラインでプロジェクト発表会を行った.

原則全員参加であるが,外部イベントと日程が被っていたこともあり参加率が例年より低かった.
これはプロジェクト発表会の日程の調整で会内で参加率の高い外部イベントを考慮しなかったことが原因であると考えられる.

プロジェクト発表会は報告書を読む時間をとった後に発表,質疑応答といった形式で行った.

発表に関しては,全ての班が予め発表スライドを作成していたため,円滑に進行した.
また,発表時間を10~20分に予め制限していたため,長時間や短時間の発表は無かった.