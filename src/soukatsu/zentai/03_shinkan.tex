\subsection*{新歓総括}

\writtenBy{\secondGrade}{西村}{雅貴}

2021年度春学期の新歓の目的は,以下の2点であった.

\begin{itemize}
    \item 新入生に会の活動内容や活動方針について知ってもらう
    \item 新入生に会に興味を持ってもらう
\end{itemize}

これらの目的を達成するため,コロナ禍の状況を鑑みて以下の3点の目標を掲げた.

\begin{itemize}
    \item 企画に対して参加してもらう
    \item 新入生に本会でやりたい事を見つけてもらう
    \item 新入生の中長期的な定着
\end{itemize}

これらの目標を達成するため,一次企画と二次企画の企画を行った.

結果的に,コロナウイルス感染拡大防止のため新歓が中止となり,これらの企画の実行も中止したが,
新歓中止の対策として大学が開催したVRSNS新歓に参加することで,新歓の目的の達成を図った.

以下に,一次企画及び二次企画の概要と,VRSNS新歓の概要と結果を示すことで,新歓総括とする.

\subsubsection*{一次企画概要}
学生オフィスが主催する新歓ブース企画に企画団体として参加し,集まった新入生に対し,会の紹介やWelcomeゼミの紹介,プロジェクト活動の成果物発表を行う事によって,目標の達成を図る.

\subsubsection*{二次企画概要}
上回生のサポートの下,新入生がProcessingを用いたゲーム制作を行うワークショップをオンラインで実施することによって,目標の達成を図る.

\subsubsection*{VRSNS新歓}
VRSNS新歓とは立命館大学学友会が主催したバーチャルブースで行う交流会である.
clusterというアプリ上で4週間行われ,そのうち6日間参加した.
ブースに来た新入生に対し,一次企画と同様に会の紹介やWelcomeゼミの紹介など行う事によって,目標の達成を図る.


結果としてVRSNS新歓自体,認知度が低かったため数人程度しか交流会に参加されなかった.よって新歓の目的は達成されなかった.
