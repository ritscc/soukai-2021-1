\subsection*{立命の家総括}

%\writtenBy{\president}{山本}{京介}
%\writtenBy{\subPresident}{山本}{京介}
%\writtenBy{\firstGrade}{山本}{京介}
\writtenBy{\secondGrade}{山本}{京介}
%\writtenBy{\thirdGrade}{山本}{京介}
%\writtenBy{\fourthGrade}{山本}{京介}

\subsubsection*{概要}
立命の家は例年8月に立命館大学で開催される小学生を対象とした企画である.
2020年度はコロナ禍のため開催されなかったが,2021年度はオンラインで開催されることが決まった.
企画にはZoomが用いられた.
2021年度は8月19日,20日に開催された.
例年は20人から40人の小学生を募集しているが,2021年度はオンライン開催であることを踏まえ,15人の小学生を募集した.
例年は各団体が2回ずつ企画を行うが,2021年度は1回ずつであった.

\subsubsection*{目的と目標}
立命の家の目的は,本会が学術部公認団体として求められる還元活動の義務を果たすことであり,
これを達成するために本企画に参加する小学生に本会の活動と情報技術に興味関心を向けてもらうことを目標としていたが,
企画に参加した小学生の反応が好ましかったことやアンケートの回答が概ね好評であったことから,この目標は達成されたと言える.

加えて,小学生に教える体験を通して会員の教える能力の向上をはかり,今後の還元活動を円滑にできるようになることを目標としていたが,
2021年度は複数の会員が教える経験ができたため,これも達成できたと言える.

\subsubsection*{役割}
2021年度は以下の様な役が割り当てられた.
\begin{itemize}
  \item 担当者
  \item 企画リーダー
  \item 企画スタッフ
\end{itemize}
担当者は例年新\secondGrade{}2人が割り当てられ,それぞれ立命の家実行委員,副実行委員となるが,
2021年度は担当者の1人が除名されたことにより副実行委員が空席となった.
これに伴い,副実行委員が兼任する企画リーダーも空席となったが,実行委員は企画に参加できないため,
2020年度の立命の家担当者の1人を割り当てることとなった.

例年は引率スタッフも割り当てているが,2021年度はオンライン開催であったため割り当てなかった.

企画スタッフは企画当日まで募集し,最終15人の会員が集まった.

\subsubsection*{企画}
2021年度は例年通りScratchを用いてプログラミング体験を実施した.
2020年度にProcessingを用いるという方針を立てたが,参加する小学生が低学年中心であったこと,オンライン開催であることを踏まえ,取りやめた.
2021年度は一つのゲームを製作してもらうことにした.
製作するゲームは事前に募集していたが,最終的に企画リーダーにより考案されたものに決定した.

企画進行は当初,企画リーダーが主に進行し,作業に詰まった小学生を企画スタッフと共にブレイクアウトルームに割り当てるという形を取る予定であった.
しかしこのままだと,理解度の差により取り残されてしまう小学生が出てくること,企画進行が属人化することが危惧されるため,
Scratchのチュートリアルだけを企画リーダーが行い,残りの時間はブレイクアウトルームに企画スタッフと小学生を割り当て,各々ゲーム製作に取り組んでもらうという形をとった.

ブレイクアウトルームにはゲーム製作を指導するスタッフ1人,それを補助するスタッフ1人,小学生2人を割り当てた.

\subsubsection*{評価点}
\paragraph{会独自リハーサルの開催}
まず挙げられるのは企画スタッフの多さである.
企画スタッフはオンライン開催ということもあり,企画進行に十分な人数が集まった.
スタッフが多いと柔軟な対応が行いやすいため,次回の立命の家もできるだけ多く集めたい.
マンツーマンの指導ができるくらいの人数が集まるのが理想である.

次にこまめな進捗確認が挙げられる.
一つのブレイクアウトルームには大体2人の小学生が割り当てられていたが,
こまめに進捗確認を行うことでどちらかの小学生が取り残されるという事態を極力減らすことができた.

次に会独自にリハーサルを開催したことが挙げられる.
立命の家では各団体が1度ずつ当日の流れを確認するために全体リハーサルを行うが,
それに加え会独自にリハーサルを開催した.
これにより当日の流れを詳しく確認したり,スライドの完成度を高めたりすることができた.

\subsubsection*{反省点}
まず挙げられるのは準備が遅れたことである.
資料が完成したのが前日であったため事前に参加する小学生に配布することができなかった.
会独自のリハーサルも前日に開催したため作業量や難易度の調整があまりできなかった.

次に企画リーダーを\thirdGrade{}が担当したことが挙げられる.
\secondGrade{}の立命の家担当者が除名されたため仕方ないことではあったが,
やはり引き継ぎの観点から\secondGrade{}が企画リーダーを担当するべきであった.

次に学年制限を設けなかったことが挙げられる.
1年生や2年生にはPCの操作自体も難しかったため,ある程度の学年制限の必要性を感じた.

次にPCが必要であることを会としては明言しなかったことが挙げられる.
今回の企画で製作するゲームはPCでの製作を前提としていたが,
iPadで参加してしまった小学生がいた.
一応立命の家全体としてはPCでの参加をお願いしているが,
会としてPCが必要であることを明言しなかったことも事実であるため反省点として挙げる.

次にカメラの使用を強制しなかったことが挙げられる.
小学生の反応を見るためにZoomのリアクション機能を用いようとしたが,
リアクション機能の使用方法で戸惑ってしまう小学生がいたり,表情がないと反応が分かりづらかったりしたため,
カメラの使用を参加条件に加えるべきであった.
