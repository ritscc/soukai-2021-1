\subsection*{サーバ管理総括}

%\writtenBy{\systemChief}{宇佐}{基史}
\writtenBy{\systemStaff}{宇佐}{基史}

2020年度秋学期総会に掲げた以下の五つからなるサーバ管理方針に従い,
2021年度春学期における活動を総括する.
\begin{itemize}
    \item Cent Streamの調査
    \item 各種ミドルウェアの更新
    \item NASの運用
    \item 管理・運用の属人化を防ぐ
    \item 自主ドメインの更新など
\end{itemize}

\subsubsection*{Cent Streamの調査}
調査の結果,
Cent Streamは保守性の点において本会の求める水準に達していないと結論付けられた.
代替案としてOracle Linux,Ubuntu Server,Ubuntu等の
他のLinuxディストリビューションの可能性も模索している.
現状はCentOS 7による稼働である.
情報の煩雑さや局員間の非公式の場での調査や知見のやり取りが行われた等の
点から一部見解や経緯が文書としてまとめられていない部分がある.

\subsubsection*{各種ミドルウェアの更新}
脆弱性情報の確認,年に複数回の最新バージョン確認は十分に行われた.
ミドルウェアの更新はリモート環境下で行われた.

\subsubsection*{NASの運用}
利用に際した方針や規則の目処が立てられず,運用に至らなかった.

\subsubsection*{管理・運用の属人化を防ぐ}
引き続きに必要となる情報の文書化が始められた.
内容の確認やその決定の経緯を局会議内で確認,共有をし,各局員の深い理解に努めた.

\subsubsection*{自主ドメインの更新など}
更新業務は適切に行われた.\\
独自ドメイン(rits.cc)の更新に伴い問題が発生したがDNSSECの設定によりこれを解決した.
原因は不明である.
