\subsection*{本会規約の改正に関する議案}
\writtenBy{\president}{深田}{紘希}

立命館コンピュータクラブ規約第二十四条に基づいて規約改正の発議を行う.

\subsubsection*{目的}
コロナ禍において部室の利用が制限制限されたり購入申請が不可能だったことにより,本年度は十分な活動ができたとはいえない.
そこで規約に記載されているように例年通り6000円の部費を集めることは道理に合わない.
よって規約を変更し,この額を可変にすることを目的とする.

\subsubsection*{現状規約}
第六章 会計

第三十八条(会費・入会金)

会費は年額六千円を会計局に納入する.また,入会金は一切徴収しない.

\subsubsection*{新規約案}
第六章 会計

第三十八条(会費・入会金)

会費は年額六千円を会計局に納入する.また,入会金は一切徴収しない.
ただし,総会もしくは臨時総会にて特例案の承認により,本年度に限り之の額を変更することが出来る.

\subsubsection*{目的}
現状,各局から欠員が発生した際にその局の運営が困難となる恐れがある.
特に三役の会計が存在する会計局の運営は必要不可欠であり,機能を停止させてはならない.
そこで会計局と総務局を統合して会計総務局とし,会計業務を滞らせないことを目的とする.

\subsubsection*{現状規約}
\subsubsection*{第十条(会計局)}
会計局は会計局長一名,会計局員数名を以って構成され,本会の会計事務に対する責任を負う.

\subsubsection*{第十四条(総務局)}
総務局は総務局長一名,総務局員数名を以って構成され,本会の備品管理,対内活動全般,会議における書記を行う.

\subsubsection*{新規約案}
\subsubsection*{第十四条(総務局)}
撤廃

\subsubsection*{第十条(会計総務局)}
会計総務局は会計総務局長一名,会計総務局員数名を以って構成され,
本会の会計事務,本会の備品管理,対内活動全般,会議における書記を行う.

会計局が会計総務局へ統合されることにより,規約の会計局の部分を会計総務局に名称変更する.
\begin{itemize}
  \item 第八条(執行部)
  \item 第九条(局体制)
  \item 第三十八条(会費・入会金)
  \item 第三十九条(会計)
\end{itemize}



